\documentclass[]{book}
\usepackage{lmodern}
\usepackage{amssymb,amsmath}
\usepackage{ifxetex,ifluatex}
\usepackage{fixltx2e} % provides \textsubscript
\ifnum 0\ifxetex 1\fi\ifluatex 1\fi=0 % if pdftex
  \usepackage[T1]{fontenc}
  \usepackage[utf8]{inputenc}
\else % if luatex or xelatex
  \ifxetex
    \usepackage{mathspec}
  \else
    \usepackage{fontspec}
  \fi
  \defaultfontfeatures{Ligatures=TeX,Scale=MatchLowercase}
\fi
% use upquote if available, for straight quotes in verbatim environments
\IfFileExists{upquote.sty}{\usepackage{upquote}}{}
% use microtype if available
\IfFileExists{microtype.sty}{%
\usepackage{microtype}
\UseMicrotypeSet[protrusion]{basicmath} % disable protrusion for tt fonts
}{}
\usepackage{hyperref}
\hypersetup{unicode=true,
            pdftitle={Urban Transportation Planning},
            pdfauthor={Gregory Macfarlane, PhD, PE},
            pdfborder={0 0 0},
            breaklinks=true}
\urlstyle{same}  % don't use monospace font for urls
\usepackage{natbib}
\bibliographystyle{apalike}
\usepackage{longtable,booktabs}
\usepackage{graphicx,grffile}
\makeatletter
\def\maxwidth{\ifdim\Gin@nat@width>\linewidth\linewidth\else\Gin@nat@width\fi}
\def\maxheight{\ifdim\Gin@nat@height>\textheight\textheight\else\Gin@nat@height\fi}
\makeatother
% Scale images if necessary, so that they will not overflow the page
% margins by default, and it is still possible to overwrite the defaults
% using explicit options in \includegraphics[width, height, ...]{}
\setkeys{Gin}{width=\maxwidth,height=\maxheight,keepaspectratio}
\IfFileExists{parskip.sty}{%
\usepackage{parskip}
}{% else
\setlength{\parindent}{0pt}
\setlength{\parskip}{6pt plus 2pt minus 1pt}
}
\setlength{\emergencystretch}{3em}  % prevent overfull lines
\providecommand{\tightlist}{%
  \setlength{\itemsep}{0pt}\setlength{\parskip}{0pt}}
\setcounter{secnumdepth}{5}
% Redefines (sub)paragraphs to behave more like sections
\ifx\paragraph\undefined\else
\let\oldparagraph\paragraph
\renewcommand{\paragraph}[1]{\oldparagraph{#1}\mbox{}}
\fi
\ifx\subparagraph\undefined\else
\let\oldsubparagraph\subparagraph
\renewcommand{\subparagraph}[1]{\oldsubparagraph{#1}\mbox{}}
\fi

%%% Use protect on footnotes to avoid problems with footnotes in titles
\let\rmarkdownfootnote\footnote%
\def\footnote{\protect\rmarkdownfootnote}

%%% Change title format to be more compact
\usepackage{titling}

% Create subtitle command for use in maketitle
\providecommand{\subtitle}[1]{
  \posttitle{
    \begin{center}\large#1\end{center}
    }
}

\setlength{\droptitle}{-2em}

  \title{Urban Transportation Planning}
    \pretitle{\vspace{\droptitle}\centering\huge}
  \posttitle{\par}
    \author{Gregory Macfarlane, PhD, PE}
    \preauthor{\centering\large\emph}
  \postauthor{\par}
      \predate{\centering\large\emph}
  \postdate{\par}
    \date{2020-04-24}

\usepackage{booktabs}

\begin{document}
\maketitle

{
\setcounter{tocdepth}{1}
\tableofcontents
}
\hypertarget{syllabus}{%
\chapter*{Foreword}\label{syllabus}}
\addcontentsline{toc}{chapter}{Foreword}

This book contains course notes and assignments for a senior / graduate class in
transportation planning and elementary travel modeling. A description for this course
is:

\begin{quote}
An advanced course in urban transportation planning. Urban transportation as the outcome of an economic system, details and techniques for four-step travel model development, applications of travel models within a legal and regulatory context.
\end{quote}

The book is organized into six units:

\begin{enumerate}
\def\labelenumi{\arabic{enumi}.}
\tightlist
\item
  \protect\hyperlink{chap-blocks}{Building Blocks}
\item
  \protect\hyperlink{chap-tripgen}{Trip Generation}
\item
  \protect\hyperlink{chap-distribution}{Trip Distribution}
\item
  \protect\hyperlink{chap-modechoice}{Mode and Destination Choice}
\item
  \protect\hyperlink{chap-assignment}{Network Assignment and Validation}
\item
  \protect\hyperlink{chap-process}{The Planning Process}
\end{enumerate}

It may seem strange to put the chapter covering the planning process at the end
of the course, after students have learned the details of quantitative travel
modeling. The purpose for this is that I assign a term project where the
students build and calibrate a four-step model as they learn the techniques to do
so, and then complete an alternatives analysis using their models. To create
the time and space to do this project, we cover ``softer'' and conceptual topics
in the second half of the course.

\hypertarget{chap-blocks}{%
\chapter{Building Blocks}\label{chap-blocks}}

\hypertarget{chap-tripgen}{%
\chapter{Trip Generation}\label{chap-tripgen}}

\hypertarget{chap-distribution}{%
\chapter{Trip Distribution}\label{chap-distribution}}

\hypertarget{chap-modechoice}{%
\chapter{Mode and Destination Choice}\label{chap-modechoice}}

\hypertarget{chap-assignment}{%
\chapter{Network Assignment and Validation}\label{chap-assignment}}

\hypertarget{chap-process}{%
\chapter{The Planning Process}\label{chap-process}}

\bibliography{book.bib}


\end{document}

\documentclass[]{book}
\usepackage{lmodern}
\usepackage{amssymb,amsmath}
\usepackage{ifxetex,ifluatex}
\usepackage{fixltx2e} % provides \textsubscript
\ifnum 0\ifxetex 1\fi\ifluatex 1\fi=0 % if pdftex
  \usepackage[T1]{fontenc}
  \usepackage[utf8]{inputenc}
\else % if luatex or xelatex
  \ifxetex
    \usepackage{mathspec}
  \else
    \usepackage{fontspec}
  \fi
  \defaultfontfeatures{Ligatures=TeX,Scale=MatchLowercase}
\fi
% use upquote if available, for straight quotes in verbatim environments
\IfFileExists{upquote.sty}{\usepackage{upquote}}{}
% use microtype if available
\IfFileExists{microtype.sty}{%
\usepackage{microtype}
\UseMicrotypeSet[protrusion]{basicmath} % disable protrusion for tt fonts
}{}
\usepackage{hyperref}
\hypersetup{unicode=true,
            pdftitle={Urban Transportation Planning},
            pdfauthor={Gregory Macfarlane, PhD, PE},
            pdfborder={0 0 0},
            breaklinks=true}
\urlstyle{same}  % don't use monospace font for urls
\usepackage{natbib}
\bibliographystyle{apalike}
\usepackage{longtable,booktabs}
\usepackage{graphicx,grffile}
\makeatletter
\def\maxwidth{\ifdim\Gin@nat@width>\linewidth\linewidth\else\Gin@nat@width\fi}
\def\maxheight{\ifdim\Gin@nat@height>\textheight\textheight\else\Gin@nat@height\fi}
\makeatother
% Scale images if necessary, so that they will not overflow the page
% margins by default, and it is still possible to overwrite the defaults
% using explicit options in \includegraphics[width, height, ...]{}
\setkeys{Gin}{width=\maxwidth,height=\maxheight,keepaspectratio}
\IfFileExists{parskip.sty}{%
\usepackage{parskip}
}{% else
\setlength{\parindent}{0pt}
\setlength{\parskip}{6pt plus 2pt minus 1pt}
}
\setlength{\emergencystretch}{3em}  % prevent overfull lines
\providecommand{\tightlist}{%
  \setlength{\itemsep}{0pt}\setlength{\parskip}{0pt}}
\setcounter{secnumdepth}{5}
% Redefines (sub)paragraphs to behave more like sections
\ifx\paragraph\undefined\else
\let\oldparagraph\paragraph
\renewcommand{\paragraph}[1]{\oldparagraph{#1}\mbox{}}
\fi
\ifx\subparagraph\undefined\else
\let\oldsubparagraph\subparagraph
\renewcommand{\subparagraph}[1]{\oldsubparagraph{#1}\mbox{}}
\fi

%%% Use protect on footnotes to avoid problems with footnotes in titles
\let\rmarkdownfootnote\footnote%
\def\footnote{\protect\rmarkdownfootnote}

%%% Change title format to be more compact
\usepackage{titling}

% Create subtitle command for use in maketitle
\providecommand{\subtitle}[1]{
  \posttitle{
    \begin{center}\large#1\end{center}
    }
}

\setlength{\droptitle}{-2em}

  \title{Urban Transportation Planning}
    \pretitle{\vspace{\droptitle}\centering\huge}
  \posttitle{\par}
    \author{Gregory Macfarlane, PhD, PE}
    \preauthor{\centering\large\emph}
  \postauthor{\par}
      \predate{\centering\large\emph}
  \postdate{\par}
    \date{2020-04-27}

\usepackage{booktabs}

\begin{document}
\maketitle

{
\setcounter{tocdepth}{1}
\tableofcontents
}
\hypertarget{syllabus}{%
\chapter*{Foreword}\label{syllabus}}
\addcontentsline{toc}{chapter}{Foreword}

This book contains course notes and assignments for a senior / graduate class in
transportation planning and elementary travel modeling. A description for this course
is:

\begin{quote}
An advanced course in urban transportation planning. Urban transportation as
the outcome of an economic system, details and techniques for four-step travel
model development, applications of travel models within a legal and regulatory
context.
\end{quote}

The book is organized into six units:

\begin{enumerate}
\def\labelenumi{\arabic{enumi}.}
\tightlist
\item
  \protect\hyperlink{chap-blocks}{Building Blocks}
\item
  \protect\hyperlink{chap-tripgen}{Trip Generation}
\item
  \protect\hyperlink{chap-distribution}{Trip Distribution}
\item
  \protect\hyperlink{chap-modechoice}{Mode and Destination Choice}
\item
  \protect\hyperlink{chap-assignment}{Network Assignment and Validation}
\item
  \protect\hyperlink{chap-process}{The Planning Process}
\end{enumerate}

It may seem strange to put the chapter covering the planning process at the end
of the course, after students have learned the details of quantitative travel
modeling. The purpose for this is that I assign a term project where the
students build and calibrate a four-step model as they learn the techniques to do
so, and then complete an alternatives analysis using their models. To create
the time and space to do this project, we cover ``softer'' and conceptual topics
in the second half of the course.

The demonstration model the students calibrate and study is a model built in the
Cube travel modeling software for the Roanoke, Virginia, metropolitan region.
The model is a relatively advanced four-step, trip-based model with only 250
zones. The limited zone size means that the entire model system runs in
approximately 15 minutes on a laptop computer. I am grateful to Virginia DOT for
allowing my students the use of this model. Directions on how to use the Roanoke
model are given in the \protect\hyperlink{app-demomodel}{Appendices}.

A handful of assignments require the students to write numerical programs or
estimate statistical models. Some guidance on using R and RStudio to accomplish
these assignments is also given in the \protect\hyperlink{app-rstudio}{Appendices}.

\hypertarget{chap-blocks}{%
\chapter{Building Blocks}\label{chap-blocks}}

This chapter contains concepts, definitions, and mathematical techniques that will
be used throughout the semester.

\hypertarget{planning-for-human-systems}{%
\section{Planning for Human Systems}\label{planning-for-human-systems}}

If you look out on any sufficiently busy road, you will see a steady stream of
vehicles passing by. Each vehicle is largely indistinguishable from the others,
and it is easy as an engineer responsible for that road to see the cars driving
by as little more than an input to a problem. But the \emph{people} inside the cars
should not be indistinguishable from each other. Each person who is driving or
riding in each of those cars has their own reasons to be driving on that road.
One person might be driving to work; one person might be trying to get home to
his or her family. Another car might hold a family going on vacation, or a group
of friends heading to a movie.

If you don't recognize that each person who travels is
If the road is congested, then the
only apparent solution is to ``fix'' the road by expanding its capacity.

\hypertarget{travel-model-building-blocks}{%
\section{Travel Model Building Blocks}\label{travel-model-building-blocks}}

\hypertarget{travel-analysis-zones}{%
\subsection{Travel Analysis Zones}\label{travel-analysis-zones}}

SE Data

\hypertarget{highway-networks}{%
\subsection{Highway Networks}\label{highway-networks}}

Functional Type

Link capacity

Free-flow speed

Centroid connectors

\hypertarget{matrices}{%
\subsection{Matrices}\label{matrices}}

Skim matrices

OD matrices

\hypertarget{statistical-and-mathematical-techniques}{%
\section{Statistical and Mathematical Techniques}\label{statistical-and-mathematical-techniques}}

\hypertarget{iterative-proportional-fitting}{%
\subsection{Iterative Proportional Fitting}\label{iterative-proportional-fitting}}

\hypertarget{numerical-optimization}{%
\subsection{Numerical Optimization}\label{numerical-optimization}}

\hypertarget{regression-analysis}{%
\subsection{Regression Analysis}\label{regression-analysis}}

\hypertarget{hw-blocks}{%
\section*{Homework}\label{hw-blocks}}
\addcontentsline{toc}{section}{Homework}

\begin{quote}
Some of these questions require a completed run of the demonstration model.
For instructions on accessing and running the model, see the \protect\hyperlink{app-demomodel}{Appendix}
\end{quote}

\begin{enumerate}
\def\labelenumi{\arabic{enumi}.}
\item
  With the TAZ layer and socioeconomic data in the demonstration model, make a
  set of choropleth maps showing: total households; household density; total jobs;
  job density; density of manufacturing vs office vs retail employment. Compare
  your maps with aerial imagery from Google Maps or OpenStreetMap. Describe the
  spatial patterns of the socioeconomic data in the model region. Identify which
  zones constitute the central business district, and identify any outlying
  employment centers.
\item
  With the highway network layer in the demonstration model, create maps
  showing: link functional type; link free flow speed; and link hourly capacity.
  Compare your maps with aerial imagery from Google Maps or OpenStreetMap.
  Identify the major freeways and principal arterials in the model region. \emph{Note}:
  you will need to run the demonstration model through the network setup step to
  calculate the capacities and append them to the link.
\item
  Find the shortest free-flow speed path along the network between two zones.
  Find the shortest distance path between the same two zones. Are the paths the
  same? Do the paths match what an online mapping service shows for a trip in the
  middle of the night?
\item
  Open the highway assignment report, which shows vehicle hours and miles
  traveled by facility type. What percent of the region's VMT occurs on freeways?
  What percent of the region's lane-miles are freeways?
\item
  Open the output highway network. Create a map of the
  highway links showing PM period level of service based on the volume to capacity
  ratios in the table below. How would you characterize traffic in Roanoke? Which
  is the worst-performing major facility?
\end{enumerate}

\hypertarget{chap-tripgen}{%
\chapter{Trip Generation}\label{chap-tripgen}}

\hypertarget{chap-distribution}{%
\chapter{Trip Distribution}\label{chap-distribution}}

\hypertarget{chap-modechoice}{%
\chapter{Mode and Destination Choice}\label{chap-modechoice}}

\hypertarget{chap-assignment}{%
\chapter{Network Assignment and Validation}\label{chap-assignment}}

\hypertarget{chap-process}{%
\chapter{The Planning Process}\label{chap-process}}

\hypertarget{appendix-appendix}{%
\appendix}


\hypertarget{app-demomodel}{%
\chapter{Demonstration Model}\label{app-demomodel}}

\hypertarget{running-the-model}{%
\section{Running the Model}\label{running-the-model}}

\hypertarget{files-and-reports}{%
\section{Files and Reports}\label{files-and-reports}}

\hypertarget{cube-tips-and-tricks}{%
\section{Cube Tips and Tricks}\label{cube-tips-and-tricks}}

\hypertarget{shortest-paths}{%
\subsection{Shortest Paths}\label{shortest-paths}}

\hypertarget{working-with-matrices}{%
\subsection{Working with Matrices}\label{working-with-matrices}}

\hypertarget{writing-custom-scripts}{%
\subsection{Writing Custom Scripts}\label{writing-custom-scripts}}

\hypertarget{app-rstudio}{%
\chapter{R and RStudio Help}\label{app-rstudio}}

Some students may not feel comfortable working in a console-based application
like RStudio. This appendix provides a basic bootcamp for Rstudio, but
cannot be a comprehensive manual on RStudio, and it certainly cannot be one for R.
Good places to get more detailed help include:

\begin{itemize}
\tightlist
\item
  R help manuals
\item
  Stack Overflow
\end{itemize}

Some of the sections in this appendix are text-based, and some contain little
more than links to YouTube videos created by me or someone else.

\hypertarget{projects-and-directories}{%
\section{Projects and Directories}\label{projects-and-directories}}

\hypertarget{r-packages}{%
\section{R Packages}\label{r-packages}}

\hypertarget{working-with-tables}{%
\section{Working with Tables}\label{working-with-tables}}

\hypertarget{graphics-with-ggplot}{%
\section{\texorpdfstring{Graphics with \texttt{ggplot}}{Graphics with ggplot}}\label{graphics-with-ggplot}}

\bibliography{book.bib}


\end{document}
